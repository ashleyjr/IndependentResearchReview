\documentclass[a4paper,11pt]{article}
\usepackage[margin=35mm]{geometry}
\setlength{\parskip}{6mm }
\usepackage{titling}
\setlength{\droptitle}{-31mm}
\setlength\parindent{0pt}
\usepackage{setspace}
\usepackage{enumitem}
\usepackage[UKenglish]{datetime}
\usepackage[hidelinks]{hyperref}
\singlespace

\begin{document}
\pagenumbering{gobble}%Only one page so no need for numbers 

%Title and our names
\title{\textbf{COMP6033: Independent Research Review\\Project Brief\\}}

\author{\textbf{Student}\\
	Ashley J. Robinson\\
	\href{mailto:ajr2g10@ecs.soton.ac.uk}{ajr2g10@ecs.soton.ac.uk}\\
\and
	\textbf{Supervisor}\\
	Professor Mark S. Nixon\\
	\href{mailto:msn@ecs.soton.ac.uk}{msn@ecs.soton.ac.uk}\\
}
\newdateformat{UKvardate}{%
\THEDAY\ \monthname[\THEMONTH], \THEYEAR}
\UKvardate


\date{10$^{th}$ February 2014\\[0cm]}
\maketitle

%Title
\begin{center}
{\large \textbf{Effective Tumor Classification\\ Exploiting Volumetric Image Analysis}}\\[6mm]
\end{center}

Classification of tumors into benign, potentially malignant and malignant classes is crucial for effective diagnosis and subsequent treatment. 
Many tumors can be further classified; for example lung cancer can be classified into two major classes known as Small Cell Lung Cancer (SCLC) and Non-Small Cell Lung Cancer (NSCLC).
These classes contain further subclasses and also a continuous measure of severity can be recorded. 
This results in a hierarchical classification problem where earlier stages prove difficult to diagnose.
Attempting to fully map this tumor space in humans will allow the application of image processing and machine learning techniques to achieve classification.  
  
To start with the review will sample popular volumetric image processing techniques with an emphasis on those with medical applications. 
Examining methods of non-intrusive 3D image capture in medicine and the possible features of tumors that could be extracted from those images will provide a basis for recommending classification methods.
The classifier architecture should be designed as such to prioritise the dual classification problem of a cancerous/non-cancerous diagnosis followed by subclasses.   

Drawing to a conclusion the work will be quantified and a detailed specification created, including a strong recommendation, for system implementation.
The implementation will follow the entire process from the provided image to tumor classification. 
Such a system would be aimed at medical practitioners as a tool which they can use to assist them in the review of captured images thereby increasing accuracy and efficiency.

%Brief

\end{document}
